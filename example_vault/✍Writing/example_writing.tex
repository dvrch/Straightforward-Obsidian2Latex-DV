\documentclass{extarticle}
% {}

% Packages de base
\usepackage[table]{xcolor}
\usepackage{hyperref}
\usepackage{graphicx}
\usepackage{subcaption} % for subfigures
\usepackage{amssymb} % need more symbols
\usepackage{titlesec} % so that we can add more subsections (using 'paragraph')
\usepackage{xcolor, soul} % for the highlighter
\usepackage{amsmath}
\usepackage{amsfonts}
\usepackage{cancel}
\usepackage{minted}
\usepackage{apacite} % apa citation style
\usepackage{caption} % to set smaller vertical spacing between two figures
\usepackage{cleveref} % for clever references
\usepackage{tcolorbox}
\usepackage{float} % to make the figures stay between the text at which they are defined
\usepackage{pdfpages}
\usepackage{totcount}
\usepackage{lipsum}
\usepackage{ragged2e} % can wrap text for tables in the tabularx environment
\usepackage{natbib} % Such that we avoid the error (`Illegal parameter number in definition of \reserved@a`) of not being able to add citations in captions
\usepackage{pdfcomment} % for popup comments in the .pdf
\usepackage{booktabs} % so that the toprule command works
\usepackage{soul} % to strikeout text using \st{}
\usepackage{twemojis} % for twemojis
\usepackage{rotating} % for rotating text on tables
\usepackage{tabularx}
\usepackage{longtable}
\usepackage{tabularray}
\usepackage{enumitem,amssymb}
\newlist{todolist}{itemize}{2}
\setlist[todolist]{label=$\square$}
\newtotcounter{citnum} %From the package documentation
\def\oldbibitem{} \let\oldbibitem=\bibitem
\def\bibitem{\stepcounter{citnum}\oldbibitem}
\setlength{\parindent}{0pt}
\usepackage[margin=0.9in]{geometry}

    \hypersetup{
    colorlinks   = true,    % Colours links instead of ugly boxes
    urlcolor     = blue,    % Colour for external hyperlinks
    linkcolor    = blue,    % Colour of internal links
    citecolor    = blue      % Colour of citations
    }
    

% Configuration principale 
% \setlength{\parindent}{0pt}{'\n    \\hypersetup{\n    colorlinks   = true,    % Colours links instead of ugly boxes\n    urlcolor     = blue,    % Colour for external hyperlinks\n    linkcolor    = blue,    % Colour of internal links\n    citecolor    = blue      % Colour of citations\n    }\n    '}

% # Configuration des listes et du document
\sethlcolor{yellow}      
\setcounter{secnumdepth}{4}
\setlength{\parskip}{7pt}
\let\oldmarginpar\marginpar
\renewcommand\marginpar[1]{\oldmarginpar{\tiny #1}}
\newcommand{\ignore}[1]{}

% CUSTOM FUNCTIONS


% =======================================
\begin{document}
\allowdisplaybreaks{}

\date{}
\author{DJONTSO Victorien}
\title{Titre du document(default value)}
\maketitle{Titre du document(default value)}



\tableofcontents{}
\newpage{}


% Start obsidian ref:
%\twemoji{warning} warning to users
\begin{enumerate}\section{Warnings to users}


% Start obsidian ref:
%\twemoji{warning} warning to users
\begin{enumerate}
\item Equations, tables, and figures should be written only inside the equation-block style (\texttt{README.md})
\item Any syntactic violation that happens inside the obsidian note will be passed into LateX. So far, I have not included any diagnostic routine.
\item When you have many embedded notes and linked notes in the note you want to convert, the algorithm searches within the vault to find them and save their paths in the \texttt{DO\_NOT\_DELETE\_\_note\_paths.txt} file. This searching process will take a few seconds (if you have many notes in your vault, and many linked mentions and embedded notes in your note), however, since the paths are now saved into this text file, any conversions you perform afterwards will be very fast.
\end{enumerate}

% End obsidian ref




\section{Development Tasks}


% Start obsidian ref:
%dev-tasks
\begin{todolist}
\item Allow user to create more complex configurations
\item Tables
\begin{todolist}
\item Fancy formatting
\end{todolist}
\item Allow the user to change settings from Obsidian, instead of Python
\end{todolist}

% End obsidian ref


\section{Formatting}




% Start obsidian ref:
%formatting
\textit{italic text}

\textbf{bold text}

\hl{highlighted text}

% End obsidian ref




\section{Itemization}

\subsection{Bullet list}

\begin{itemize}
\item Item 1
\begin{itemize}
\item item 1.1
\item item 1.2
\begin{itemize}
\item item 1.2.1
\end{itemize}
\end{itemize}
\item Item 2
\begin{enumerate}
\item Enumeration 1
\begin{enumerate}
\item Enumeration 1.2
\item Enumeration 2.2
\end{enumerate}
\item Enumeration 2
\begin{enumerate}
\item Enumeration 2.1
\begin{itemize}
\item Bullet 2.1.1
\end{itemize}
\end{enumerate}
\item Enumeration 3
\end{enumerate}
\end{itemize}


\subsection{Enumerated list}

\begin{enumerate}
\item Item 1
\item Item 2
\begin{enumerate}
\item Item 2.1
\item Item 2.2
\end{enumerate}
\end{enumerate}


\subsection{Task list}

\begin{todolist}
\item Task 1
\item Task 2
\item Task 3
\begin{todolist}
\item Task 3.1
\end{todolist}
\item Task 4
\end{todolist}
\section{Adding citations}\label{sec:Adding-citations}

Command: just mention that link that pertains to the literature file. I use the "p"+"number" naming convention. For example, "p1" would be the first literature file in my vault.



Example: In \${p1}, we see that...\hypertarget{ad3b86}{}



\section{Equations}

Both equations and subfigures are written in the form of embedded notes, since they are encoded as notes.

\begin{tcolorbox}[width=1.0\textwidth,colback={red},title={warning},outer arc=0mm,colupper=white]

If you write an equation outside of the designated template in an embedded note, then the conversion will be faulty!

\end{tcolorbox}

\subsection{Writing the equation}



Steps:



\begin{enumerate}
\item Press ctrl+P, then Quickadd: equation\_block\_single
\end{enumerate}


!\Cref{eq:Einstein\\\\#expr}



It supports the aligned equations, as seen in\Cref{eq:1}.



!\Cref{eq:1\\\\#expr}

\subsection{Referencing the equation}

In\Cref{eq:Einstein}, we see that...

\section{Figures}

\subsection{Adding figures}

\subsubsection{No subfigures}

!\Cref{fig:gradient_steps\\\\#fig}



![[DV Logo (128 carré ⬜ ) 1.jpg]]

\subsubsection{With subfigures}

See\Cref{fig:1}.





\begin{itemize}
\item\twemoji{plus}  Allow user to create more complex configurations
\end{itemize}
!\Cref{fig:1\\\\#fig}





\subsection{Referencing figures}

In\Cref{fig:1}, we can notice that...





\section{Admonition blocks}

If you write admonition blocks, they are translated into something similar in latex.

\textbf{Example}

\begin{tcolorbox}[width=1.0\textwidth,colback={red},title={warning},outer arc=0mm,colupper=white]

This is a warning

\end{tcolorbox}



\begin{tcolorbox}[width=1.0\textwidth,colback={white},title={note},outer arc=0mm,colupper=black]

This is a note

\end{tcolorbox}





\section{Code blocks}

\begin{minted}{python}

print("this is a code block")

print("this is another code block")

\end{minted}



\section{Cross-reference of section}

Check\hyperref[sec:Adding-citations]{this section} (\autoref{sec:Adding-citations}) about adding citations.





\section{\maltese Cross-reference of block}

example



\section{Assume sections from embedded notes}

The sections from embedded notes can assume the hierarchy of the file wherein they are embedded.

\begin{tcolorbox}[width=1.0\textwidth,colback={white},title={note},outer arc=0mm,colupper=black]

Notice in the latex file that the section hierarchy has been modified to adhere to the hierarchy of the file that embeds the note.

\end{tcolorbox}




% Start obsidian ref:
%embedded with sections
\subsubsection{Section 1 of embedded By me}
\paragraph{subsection 1 of embedded}\hspace{0pt} \\
% End obsidian ref




\section{Hyperlinks}

Click [here](\url{https://www.youtube.com/}).





\section{Tables}



See\Cref{tab:1},\Cref{tab:2}, and\Cref{tab:long}.



!\Cref{tab:1\\\\#table}



!\Cref{tab:2\\\\#table}



!\Cref{tab:long\\\\#table}
